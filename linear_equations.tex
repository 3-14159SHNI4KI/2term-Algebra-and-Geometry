\subsection*{Определение системы линейных уравнений}
\begin{definition}
    \textit{Системой линейных уравнений} (далее СЛУ) называют систему вида:
    \begin{equation}
        \label{eq:SLU}
        \begin{cases}
        a_1^1x_1 +  a_2^1x_2 + \ldots + a_n^1x_n = b_1 \\
        a_1^2x_1 +  a_2^2x_2 + \ldots + a_n^2x_n = b_2 \\
        ~\vdots \\
        a_1^mx_1 +  a_2^mx_2 + \dots + a_n^mx_n = b_m,
    \end{cases}
    \end{equation} 
    где $a_j^i \in \Real$, $b_j \in \Real$, $m$ "--- количество уравнений, $n$ "--- количество переменных, $x_j^i$ "--- неизвестные, $a_j^i$ "--- коэффициенты, $b_j$ "--- свободные члены.
\end{definition}
$A = \begin{pmatrix}
    a_1^1 & \cdots & a_n^1 \\
    \vdots & \ddots & \vdots \\
    a_1^m & \cdots & a_n^m
\end{pmatrix}$ называется \textit{матрицей коэффициентов}

$\tilde{A} = (A~|~B) =
\left({ 
\begin{array}{ccc|c}
    a_1^1 & \cdots & a_n^1 & b_1 \\
    \vdots & \ddots & \vdots & \vdots \\
    a_1^m & \cdots & a_n^m & b_m
\end{array}} \right)$ "--- \textit{расширенная матрица коэффициентов}.

Пусть $B = \begin{pmatrix}
    b_1 \\
    \vdots \\
    b_m
\end{pmatrix}$, $X = \begin{pmatrix}
    x_1 \\
    \vdots \\
    x_m
\end{pmatrix}$. 

Тогда СЛУ можно переписать в матричном виде:
$ AX = B$.

\vspace{0.2cm}

\begin{tikzpicture}
    [sibling distance=5cm,
    align=center]
     \node {СЛУ}
         child {node {совместные\\(есть решение)}
          child{node {определённые\\(единственное решение)}}
          child{node {неопределённые\\(более одного решения)}}}
         child {node {несовместные\\(нет решений)}};
\end{tikzpicture}

\subsection*{Элементарные преобразования СЛУ и строк матрицы}

\begin{definition}
    \textit{Элементарными преобразованиями СЛУ} называются преобразования вида:
    \begin{enumerate}
        \item Прибавление к одному уравнению другого, умноженного на число
        \item Перестановка двух уравнений
        \item Умножение уравнения на число, отличное от нуля
    \end{enumerate}
\end{definition}

\subsection*{Метод Гаусса}

\begin{definition}
    \textit{Ведущим} элементом ненулевой строки называется её первый ненулевой элемент
\end{definition}
\begin{definition}
    Матрица называется \textit{ступенчатой}, если
    \begin{enumerate}
        \item Номера ведущих элементов её ненулевых строк образует строго возрастающую последовательность,
        \item Ненулевые строки (если они есть) стоят в конце.
    \end{enumerate}
\end{definition}
\begin{example}[Ступенчатой матрицы]
    $$\newcommand*{\temp}{\multicolumn{1}{c|}{0}}
    \left(\begin{array}{cccccc}
        0 & \temp & \ast & \ast & \ast & \ast \\ \cline{3-4}
        0 & 0 & 0 & \temp & \ast  & \ast \\ \cline{5-5}
        0 & 0 & 0 & 0 & \temp & \ast \\ \cline{6-6}
        0 & 0 & 0 & 0 & 0 & 0
        
    \end{array}\right)
    $$
\end{example}

\begin{theorem}
    Всякую матрицу путём элементарных преобразований можно привести к ступенчатому виду.
\end{theorem}
\begin{definition}
    СЛУ называется \textit{ступенчатой}, если её расширенная матрица ступенчатая
\end{definition}

\subsection*{Система однородных линейных уравнений (СОЛУ)}
\begin{definition}
    Система линейных уравнений называется однородной, если все свободные члены уравнений равны нулю:
    \begin{equation}
        \label{eq:SOLU}
        \begin{cases}
            a_1^1x_1 +  a_2^1x_2 + \ldots + a_n^1x_n = 0 \\
            a_1^2x_1 +  a_2^2x_2 + \ldots + a_n^2x_n = 0 \\
            ~\vdots \\
            a_1^mx_1 +  a_2^mx_2 + \dots + a_n^mx_n = 0.
        \end{cases}
        \end{equation}
\end{definition}
Однородная система всегда совместна, поскольку она всегда имеет тривиальное (нулевое) решение.

\begin{theorem}
    Совокупность всех решений системы однородных линейных уравнений с $n$ неизвестными является подпространством $\Real^n$ (в общем случае $K^n$).

    Совокупность всех решений произвольной совместной СЛУ есть сумма какого"=либо одного решения и подпространства решений системы однородных линейных уравнений с той же матрицей коэффициентов.
\end{theorem}
\begin{Proof}
    Пусть $\bar{u} = (u_1, \ldots, u_n)$ и $\bar{v} = (v_1, \ldots, v_n)$ "--- решения \ref{eq:SOLU}.

    Тогда $\bar{u} + \bar{v}$ "--- решение \ref{eq:SOLU}, так как
    \begin{equation*}
        \overbrace{a_1^1u_1 + \ldots + a_n^1u_n}^0 + \overbrace{a_1^1v_1 + \ldots + a_n^1v_n}^0 = 0.
    \end{equation*}
    Также $\lambda\bar{u}$ "--- решение \ref{eq:SOLU}, так как 
    \begin{equation*}
        \lambda\overbrace{(a_1^1u_1 + \ldots + a_n^1u_n)}^0 = 0.
    \end{equation*}

    Следовательно совокупность всех решений СОЛУ с $n$ неизвестными есть подпространство $\Real^n$.

    Пусть $\bar{t} = (t_1, \ldots, t_n)$ "--- решение \ref{eq:SLU}, $\bar{u} = (u_1, \ldots, u_n)$ "--- решение \ref{eq:SOLU}.
    
    Тогда $\bar{t} + \bar{u}$ "--- решение \ref{eq:SLU}, так как
    \begin{gather*}
        a_1^1(t_1 + u_1) + \ldots + a_n^1(t_n + u_n) = b_1 \\
        \underbrace{a_1^1t_1 + \ldots + a_n^1t_n}_{b_1} + \underbrace{a_1^1u_1 + \ldots + a_n^1u_n}_0 = b_1.
    \end{gather*}

    Пусть $\bar{f}$ "--- решение \ref{eq:SLU}. Тогда $\bar{p} = \bar{t} - \bar{f}$ "--- решение \ref{eq:SOLU}, так как
    \begin{gather*}
        a_1^1(t_1 - f_1) + \ldots + a_n^1(t_n - f_n) = \\ = \underbrace{a_1^1t_1 + \ldots + a_n^1t_n}_{b_1} - \underbrace{(a_1^1f_1 + \ldots + a_n^1f_n)}_{b_1} = 0
    \end{gather*}

    Значит совокупность всех решений СЛУ есть сумма какого"=либо одного решения и подпространства СОЛУ с той же матрицей коэффициентов.
\end{Proof}

\subsection*{Векторное пространтсво}

\begin{definition}
    \textit{Векторным пространством} над полем $K$ называется множество $V$ с операциями сложения и умножения на элемент поля $K$, обладающее следующими свойствами:
    \begin{enumerate}
        \item $V$ "--- аддитивная абелева группа.
    \item $\lambda(\bar{a} + \bar{b}) = \lambda \bar{a} + \lambda \bar{b}$
        \item $(\lambda + \mu)\bar{a} = \lambda \bar{a} + \mu \bar{a}$
        \item $\lambda(\mu \bar{a}) = (\lambda \mu)\bar{a}$
        \item $1 \cdot \bar{a} = \bar{a}$
    \end{enumerate}
    $\lambda,\mu \in K,~\bar{a},\bar{b} \in V$.
\end{definition}

\begin{definition}
    Пусть $S \subset V$. Совокупность всевозможных линейных комбинаций из векторов $S$ называется \textit{линейной оболочкой} множества $S$. Обозначние: $<\!S\!>$.
\end{definition}
Пространство $V$ \textit{порождается} множеством $S$, если $V = <\!S\!>$.

\begin{definition}
    Векторное пространство называется \textit{конечномерным}, если оно порождается конечным числом векторов и бесконечномерным в противном случае.
\end{definition}
\begin{definition}
    \textit{Базисом} векторного пространства называется упорядоченная максимально линейно независимая система векторов, где каждый вектор пространства линейно выражается через эту систему.    
\end{definition}

\subsection*{Ранг}
\begin{definition}
    Рангом \textit{системы векторов} называется размерность её линейной оболочки.

    Ранг \textit{матрицы} "--- это ранг системы её строк.
\end{definition}
\begin{definition}
    Системы векторов $\{\bar{a}_1, \ldots, \bar{a}_n\}$ и $\{\bar{b}_1, \ldots, \bar{b}_m\}$ называются \textit{эквивалентными}, если каждый из векторов $\bar{b}_i$ линейно выражается через $\bar{a}_1, \ldots, \bar{a}_n$ и наоборот,
    $\bar{a}_j$ линейно выражается через $\bar{b}_1, \ldots, \bar{b}_m$, где $i = 1,\ldots,m$ и $j = 1,\ldots, n$.

    Это равносильно $<\!\bar{a}_1, \ldots, \bar{a}_n\!> = <\!\bar{b}_1, \ldots, \bar{b}_n\!>$.
\end{definition}

\begin{theorem}
    Ранг матрицы равен числу ненулевых строк ступенчатой матрицы, к которой она приводится элементарными преобразованиями строк.
\end{theorem}

Линейная зависимость между столбцами матрицы не меняется при элементарных преобразований строк. Ранг системы её столбцов не меняется при элементарных преобразований строк.

Ранг системы строк любой матрицы равен рангу системы столбцов этой матрицы.

\begin{theorem}[Кронекера-Капелли]
    СЛУ совместна $\Leftrightarrow$  $rk(A|B) = rk\;A$.
\end{theorem}
\begin{Proof}
$\Rightarrow$ Пусть система \ref{eq:SLU} совместна и $(k_1,\ldots, k_n)$ "--- решение СЛУ.
\begin{equation*}
    \begin{pmatrix}
        a_1^1 \\
        a_1^2 \\
        \vdots \\
        a_1^m
    \end{pmatrix} k_1 +
    \begin{pmatrix}
        a_2^1 \\
        a_2^2 \\
        \vdots \\
        a_2^m
    \end{pmatrix} k_2 + \ldots + 
    \begin{pmatrix}
        a_n^1 \\
        a_n^2 \\
        \vdots \\
        a_n^m
    \end{pmatrix} k_n = 
    \begin{pmatrix}
        b_1 \\
        b_2 \\
        \vdots \\
        b_m
    \end{pmatrix}
\end{equation*}
Последний столбец расширенной матрицы есть линейная комбинация остальных столбцов с коэффициентами $(k_1,\ldots,k_n)$. Всякий другой столбец расширенной матрицы входит в матрицу $A$, поэтому линейно выражается через столбцы этой матрицы. Значит система столбцов расширенной матрицы $A|B$ и $A$ эквивалентны  $\implies ~ rk(A|B) =  rk(A)$.

$\Leftarrow$ $rk(A|B) = rk(A)\implies$ максимальная линейно независимая система столбцов $A$ остаётся линейно независимой и в $A|B\implies$ через эту систему, а значит через систему столбцов матрицы $A$, линейно выражается последний столбец $A|B\implies\exists(k_1,\ldots,k_n)$:
\begin{equation*}
    \begin{pmatrix}
        a_1^1 \\
        a_1^2 \\
        \vdots \\
        a_1^m
    \end{pmatrix} k_1 +
    \begin{pmatrix}
        a_2^1 \\
        a_2^2 \\
        \vdots \\
        a_2^m
    \end{pmatrix} k_2 + \ldots + 
    \begin{pmatrix}
        a_n^1 \\
        a_n^2 \\
        \vdots \\
        a_n^m
    \end{pmatrix} k_n = 
    \begin{pmatrix}
        b_1 \\
        b_2 \\
        \vdots \\
        b_m
    \end{pmatrix}
\end{equation*}
то есть $(k_1,\ldots, k_n)$ "--- решение СЛУ $\implies$ СЛУ совместна.
\end{Proof}

\begin{theorem}
    Совместная СЛУ является определённой $\Leftrightarrow$ $rk(A)$ равен числу неизвестных.
\end{theorem}

\begin{theorem}
    Размерность пространства решений СОЛУ с $n$ неизвестными и матрицей коэффициентов $A$ равняется $n - rk\;A$.
\end{theorem}
\begin{Proof}
    Приводим \hyperref[eq:SOLU]{СОЛУ} путём элементарных преобразований к ступенчатому виду.

    Число ненулевых уравнений в ступенчатом виде равно $r = rk(A)$. 
    Общее решение системы содержит $n$ переменных:
    $$
    \begin{cases}
        x_1 = c_1^1x_{r + 1} + c_2^1x_{r + 2} + \ldots + c_{n - r}^1x_n \\
        x_2 = c_1^2x_{r + 1} + c_2^2x_{r + 2} + \ldots + c_{n - r}^2x_n \\
        ~\vdots \\
        x_r = c_1^rx_{r + 1} + c_2^rx_{r + 2} + \ldots + c_{n - r}^rx_n.
    \end{cases}
    $$
\end{Proof}
Частные решения:

Придавая по очереди одному из свободных неизвестных $x_{r + 1}, \ldots, x_n$ значения $1$, а остальным значения $0$, получим решение системы:
$$
\begin{array}{l}
    \bar{u}_1 = (c_1^1,c_1^2,\ldots,c_1^r,1,0,\ldots,0) \\
    \bar{u}_2 = (c_2^1,c_2^2,\ldots,c_2^r,0,1,\ldots,0) \\
    \vdots \\
    \bar{u}_{n - r} = (c_{n - r}^1, c_{n - r}^2, \ldots, c_{n - r}^r, 0, 0, \ldots, 1).
\end{array}
$$
Докажем, что $\bar{u}_1, \ldots, \bar{u}_{n - r}$ образуют базис пространства решений (это есть ядро линейного отображения см. определение \ref{def:core_lin}).
\begin{Proof}
    $\bar{u} = \lambda_1\bar{u}_1 + \ldots + \lambda_{n - r}\bar{u}_{n - r}$, $\bar{u}$ "--- решение системы.

    $\bar{0} = \lambda_1\bar{u}_1 + \ldots + \lambda_{n - r}\bar{u}_{n - r} \Leftrightarrow \lambda_1 = \ldots = \lambda_{n - r} = 0$
    $\implies \bar{u}_1, \ldots, \bar{u}_{n - r}$ "--- линейно независимы.
    
    Значит $(\bar{u}_1, \ldots, \bar{u}_{n - r})$ "--- базис пространства решений. 
\end{Proof}

\begin{definition}
    Всякий базис пространства решений СОЛУ называется \textit{фундаментальной} системой решения.
\end{definition}

\subsection*{Линейные отображения}
\begin{definition}
    Пусть $U,\;V$ "--- действительные векторные пространства.
    \textit{Линейным отображением} называется отображение $\varphi: U \to V$, удовлетворяющее свойствам:
    \begin{enumerate}
        \item $\varphi(\bar{a} + \bar{b}) = \varphi(\bar{a}) + \varphi(\bar{b})$
        \item $\varphi(\lambda\bar{a}) = \lambda\varphi(\bar{a})$,
    \end{enumerate}
    $\lambda \in \Real$, $\bar{a},\;\bar{b} \in U$
    
\end{definition}
В таком случае векторные пространства $U$ и $V$ называются \textit{изоморфными}. 

Если $U$ совпадает c $V$, то $\varphi$ называется \textit{линейным преобразованием}.

Вспомнив запись СЛУ через матрицы, получаем, что 
$\varphi: \Real^n \to \Real^m$ и $\varphi(\bar{x}) = \bar{b}$, где $\bar{x} = (x_1, \ldots, x_n)$ "--- решение СЛУ.

Пусть  $A$ "--- матрица коэффициентов, тогда $A_\varphi$ "--- \textit{матрица линейного отображения}.

Для СОЛУ: $\varphi(\bar{x}) = \bar{0}$,где $\bar{x} = (x_1, \ldots, x_n)$ "--- решение СОЛУ.
\begin{definition}
    \textit{Образом} линейного отображения $\varphi: U \to V$ называется множество $Im\;\varphi = \{\bar{b} \in V, ~\varphi(\bar{x}) = \bar{b}, ~\bar{x} \in U \}$.
\end{definition}
\begin{definition}
    \label{def:core_lin}
    \textit{Ядром} линейного отображения $\varphi: U \to V$ называется множество $Ker\;\varphi = \{\bar{x} \in U:~\varphi({\bar{x}}) = \bar{0} \}$.
\end{definition}
Заметим, что $Ker\;\varphi \subset U$ и $Im\;\varphi \subset V$.

\begin{theorem}
    $\varphi(\bar{a}) = \bar{b}$, $\varphi(\bar{x}) = \bar{b}$

    $\implies \bar{x} = \bar{a} + Ker\;\varphi$.
\end{theorem}


